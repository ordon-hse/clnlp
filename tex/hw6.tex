\documentclass[12pt]{article}
\usepackage[utf8]{inputenc}
\usepackage[english]{babel}
\usepackage{amsmath}
\usepackage{amsthm}
\usepackage{hyperref}
\usepackage{xcolor}
\usepackage{graphicx}

\renewcommand{\baselinestretch}{1.} 
\renewcommand{\familydefault}{\sfdefault}

\begin{document}
\title{Probability Theory and Mathematical Statistics. HW5}
\author{O.R.Don}
\maketitle

\section*{Part I}

\begin{itemize}
\item[1.]
Using the goodness-of-fit test procedure based on the chi-square distribution and considering the $\alpha = 0.05$, test: \\
The form of the distribution of the number of bombs that fell on one district is Poisson. \\
$H_0 : X\sim Pois(\theta).$ \\
The form of the distribution of the number of bombs that fell on one district is not Poisson. \\
$H_1 : X\not\sim Pois(\theta). $ \\

\textit{Solution.} $H_0 : X\sim Pois(\theta), H_1 : X\not\sim Pois(\theta), \alpha = 0.05, k = 6, l = 1. \\
\gamma = 1 - 0.25 = 0.975 \Rightarrow G = [\chi_{0.025}^2(4); \chi_{0.975}(4)] = [0.4844; 11.1433]. \\
\hat\theta = \bar{x} = 0.9288. \\
z = \sum\limits_{i=1}^k\frac{(v_i - np_i(\hat\theta))^2}{np_i(\hat\theta)} = 1.0544\in G \Rightarrow \text{ accept } H_0. \\
$

\textbf{Answer: accept $H_0$.}

\item[2.]
Using the goodness-of-fit test procedure based on the chi-square distribution and considering the $\alpha = 0.05$, test: \\
The form of the distribution of the men height is normal. \\
$H_0 : X\sim N(\theta_1, \theta_2^2).$ \\
The form of the distribution of the men height is not normal. \\
$H_1 : X\not\sim N(\theta_1, \theta_2^2). $ \\

\textit{Solution.} $H_0 : X\sim N(\theta_1, \theta_2^2), H_1 : X\not\sim N(\theta_1, \theta_2^2), \alpha = 0.05, k = 15, \\
l = 2. \\
\gamma = 1 - 0.25 = 0.975 \Rightarrow G = [\chi_{0.025}^2(12); \chi_{0.975}(12)] = [4.4038; 23.3367]. \\
\hat\theta_1 = \bar{x} = 165.533, \hat\theta_2 = s = 6.053. \\
$
After that, since we have intervals as x values, we take CDF of ends of intervals ($x_i, x_{i+1}$) and differences between their CDF values ($p_i = F(x_{i+1}) - F(x_i)$). \\
$z = \sum\limits_{i=1}^{k}\frac{(v_i - np_i(\hat\theta_1, \hat\theta_2))^2}{np_i(\hat\theta_1, \hat\theta_2)} = 2.0803 \not\in G \Rightarrow \text{ reject } H_0. \\
$

\textbf{Answer: reject $H_0$.}

\item[3.]
Using the contingency table tests procedure based on the chi-square distribution and considering the $\alpha = 0.05$, test: \\
$H_0 : X\text{ is independent of }Y. \\
H_1 : X\text{ is depends on }Y. \\
$

\textit{Solution.} $H_0 : X\text{ is independent of }Y, H_1 : X\text{ is depends on }Y. \\
\alpha = 0.05 \Rightarrow G = (-\infty; \chi_{0.05}^2(4\cdot 3)] = (-\infty; 7.9662]. \\
z = n(\sum\limits_{i=1}^n\sum\limits_{j=1}^m \frac{v_{ij}^2}{v_{x,i}v_{y,j}} - 1) = -479428 \in G \Rightarrow \text{ accept } H_0. \\
$

\textbf{Answer: accept $H_0$.}

\item[4.]
Using the contingency table tests procedure based on the chi-square distribution and considering the $\alpha = 0.05$, test: \\
$H_0 : X\text{ is independent of }Y. \\
H_1 : X\text{ is depends on }Y. \\
$

\textit{Solution.} $H_0 : X\text{ is independent of }Y, H_1 : X\text{ is depends on }Y. \\
\alpha = 0.05 \Rightarrow G = (-\infty; \chi_{0.05}^2(5\cdot 1)] = (-\infty; 3.9403]. \\
z = n(\sum\limits_{i=1}^n\sum\limits_{j=1}^m \frac{v_{ij}^2}{v_{x,i}v_{y,j}} - 1) = -2026514 \in G \Rightarrow \text{ accept } H_0. \\
$

\textbf{Answer: accept $H_0$.}

\item[5.]
Suppose that 10 sets of hypotheses of the form: \\
$H_0 : \mu_X = \mu_0 \text{(const)} \\
H_1 : \mu_X = \mu_0 \text{(const)} \\
$
Have been tested and the P-values for these tests are 0.12, 0.08, 0.93, 0.02, 0.01, 0.05, 0.88, 0.15, 0.13 and 0.06. Use Fisher's procedure to combine all of these P-values. What conclusions can you draw about these hypotheses?

\textit{Solution.} $\alpha = 0.05, n = 10, p_i = i^{th}\text{ P-value}. \\
\gamma = 0.975 \Rightarrow G = [\chi_{0.25}^2(20);\chi_{0.975}^2(20)] = [9.5908; 34.1696]. \\
z = -2\sum\limits_{i=1}^n \ln p_i = 46.2201 \not\in G \Rightarrow \\
\Rightarrow \text{ according to Fisher's combined probability test we reject } H_0.
$

\textbf{Answer: reject $H_0$.}

\end{itemize}

\end{document}